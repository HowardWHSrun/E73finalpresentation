\documentclass[aspectratio=169]{beamer}

\usetheme{Madrid}
\usecolortheme{beaver}

\title{Physical Electronics \& Architecture: The ARM Approach}
\subtitle{From Silicon Atoms to System-on-Chip}
\author{Engineering 73}
\date{\today}

\begin{document}

\begin{frame}
    \titlepage
\end{frame}

\section{The Physics Foundation (Bottom-Up)}

\begin{frame}{The Fundamental Unit: The MOSFET}
    \textbf{Concept:} Every processor starts with the Metal-Oxide-Semiconductor Field-Effect Transistor (MOSFET). It is a voltage-controlled switch.
    
    \vspace{0.5cm}
    
    \textbf{The Physics:}
    \begin{itemize}
        \item \textbf{Gate:} The control knob.
        \item \textbf{Channel:} The path for electrons (Source $\to$ Drain).
        \item \textbf{Mechanism:} Applying voltage to the Gate creates an electric field, allowing current to flow.
    \end{itemize}
\end{frame}

\begin{frame}{The Governing Equation (Current)}
    \textbf{Concept:} The current ($I_D$) flowing through the transistor determines how fast it switches.
    
    \begin{equation*}
        I_D \approx \frac{1}{2} \mu C_{ox} \frac{W}{L} (V_{GS} - V_{TH})^2
    \end{equation*}
    
    \begin{itemize}
        \item $\frac{W}{L}$ (Geometry): The physical width and length of the transistor.
        \item $V_{GS} - V_{TH}$ (Overdrive Voltage): The "gas pedal."
    \end{itemize}
    
    \vspace{0.5cm}
    \textbf{Key Takeaway:} To make chips faster, we usually increase $W/L$ (size) or $V_{GS}$ (voltage). But this costs power and space.
\end{frame}

\begin{frame}{From Physics to Logic (CMOS)}
    \textbf{Concept:} We don't use single transistors; we use Complementary MOS (CMOS) to build Logic Gates.
    
    \vspace{0.5cm}
    
    \textbf{The Setup:}
    \begin{itemize}
        \item \textbf{NMOS:} Pulls voltage down (Logic 0).
        \item \textbf{PMOS:} Pulls voltage up (Logic 1).
    \end{itemize}
    
    \vspace{0.5cm}
    \textbf{The Link:} A NAND gate (universal logic) requires 4 transistors. A CPU is just billions of these arranged specifically.
\end{frame}

\section{The Architectural Bridge}

\begin{frame}{The Challenge: Complexity vs. Efficiency}
    \textbf{The Question:} We have the physics (MOSFETs). How do we arrange them?
    
    \vspace{0.5cm}
    
    \textbf{Two Schools of Thought:}
    \begin{itemize}
        \item \textbf{CISC (x86):} Complex hardware to do heavy lifting. Large physical footprint.
        \item \textbf{RISC (ARM):} Simple hardware. Shift complexity to software. Small physical footprint.
    \end{itemize}
    
    \vspace{0.5cm}
    \textbf{Thesis:} ARM is a blueprint for Physical Minimalism.
\end{frame}

\begin{frame}{The Physical Consequence of RISC}
    \textbf{Concept:} ARM uses Simple, Fixed-Length Instructions.
    
    \vspace{0.5cm}
    
    \textbf{Physical Impact:}
    \begin{itemize}
        \item \textbf{Decoding:} x86 requires massive, power-hungry circuits just to figure out what an instruction is.
        \item \textbf{ARM:} The decoder is tiny and simple because the instructions are uniform.
    \end{itemize}
    
    \vspace{0.5cm}
    \textbf{Result:} Fewer transistors needed for "overhead" logic $\to$ Lower Capacitive Load ($C_{load}$).
\end{frame}

\section{The Governing Equations of Efficiency}

\begin{frame}{The Master Equation (Dynamic Power)}
    \textbf{Concept:} Why does ARM battery life last longer? It's in the math.
    
    \begin{equation*}
        P_{dynamic} \approx C_{load} \cdot V_{DD}^2 \cdot f
    \end{equation*}
    
    \textbf{The Breakdown:}
    \begin{itemize}
        \item $C_{load}$: Reduced by ARM's simple RISC logic (fewer transistors switching).
        \item $V_{DD}$ (Voltage): The most critical term (squared!). ARM designs are optimized to run at lower voltages.
    \end{itemize}
\end{frame}

\begin{frame}{Speed via Structure (Pipelining)}
    \textbf{Concept:} If individual transistors are running at low power, how do we get speed? Pipelining.
    
    \begin{equation*}
        \text{Execution Time} = \text{CPI} \cdot \text{Clock Cycle} \cdot \text{Instruction Count}
    \end{equation*}
    
    \textbf{The ARM Strategy:}
    \begin{itemize}
        \item \textbf{CPI (Cycles Per Instruction):} Because instructions are simple, ARM achieves a CPI $\approx 1$.
        \item \textbf{Pipeline:} Uniform instructions allow for a smooth, jam-free assembly line (Fetch $\to$ Decode $\to$ Execute).
    \end{itemize}
\end{frame}

\begin{frame}{Physical Gate Delay (The Speed Limit)}
    \textbf{Concept:} What stops us from going faster? The physics of the transistor again.
    
    \begin{equation*}
        \tau_{pd} \propto \frac{C_{out} \cdot V_{DD}}{(V_{DD} - V_t)^\alpha}
    \end{equation*}
    
    \textbf{The Trade-off:} ARM accepts a slightly higher delay ($\tau_{pd}$)—meaning lower peak frequency ($f$)—in exchange for a massive drop in Voltage ($V_{DD}$), optimizing the Power-Per-Watt ratio.
\end{frame}

\section{The Result (Silicon Reality)}

\begin{frame}{Area \& Cost (The Economic Equation)}
    \textbf{Concept:} Smaller cores cost less and allow for more integration.
    
    \begin{equation*}
        \text{Cost} \propto \text{Die Area}^2
    \end{equation*}
    
    \textbf{Impact:} Because ARM cores are physically small, we can fit 8 Cores + GPU + Neural Engine on one small piece of silicon.
\end{frame}

\begin{frame}{System-on-Chip (SoC)}
    \textbf{Concept:} The ARM Core is just one part of the physical electronics.
    
    \vspace{0.5cm}
    
    \textbf{The Map:}
    \begin{itemize}
        \item Core (ALU/Registers) connects to:
        \item Memory Controller
        \item Physical I/O
    \end{itemize}
\end{frame}

\begin{frame}{Conclusion}
    \textbf{Summary:}
    \begin{itemize}
        \item \textbf{Physics:} Transistors follow the Square Law.
        \item \textbf{Architecture:} ARM's RISC design minimizes the transistor count ($C_{load}$).
        \item \textbf{Result:} Lower Power ($P \propto V^2$), Smaller Area, and High Efficiency.
    \end{itemize}
    
    \vspace{1cm}
    \centering
    \textit{ARM isn't just code; it's a specific strategy for organizing silicon atoms.}
\end{frame}

\end{document}
