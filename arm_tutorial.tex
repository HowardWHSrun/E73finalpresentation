\documentclass[12pt]{article}

\usepackage[margin=1in]{geometry}
\usepackage[T1]{fontenc}
\usepackage{lmodern}
\usepackage{microtype}
\usepackage{hyperref}
\usepackage{amsmath,amssymb}
\usepackage{xcolor}
\usepackage{booktabs}

% Removed enumitem to avoid compilation errors
\setlength{\parskip}{6pt}
\setlength{\parindent}{0pt}

\hypersetup{
  colorlinks=true,
  linkcolor=blue,
  urlcolor=blue,
  citecolor=blue
}

% --- Small helpers for readability ---
\newcommand{\Slide}[1]{\subsection*{Slide #1}}
\newcommand{\Goal}[1]{\textbf{Goal:} #1\\}
\newcommand{\PointAt}[1]{\textbf{Point at:} #1\\}
\newcommand{\Say}{\textbf{Say (script):}}
\newcommand{\IfAsked}{\textbf{If asked (deeper):}}
\newcommand{\Transition}[1]{\textbf{Transition line:} #1}
\newcommand{\KeyIdea}[1]{\textbf{Key idea:} #1}

\title{The ARM Advantage\\\large Printable Presenter Tutorial (Detailed)}
\author{Engineering 73}
\date{\today}

\begin{document}
\maketitle
\tableofcontents
\newpage

\section{How to use this tutorial}
This document is designed so you can (1) truly understand the material, (2) rehearse efficiently, and (3) handle Q\&A confidently.

Use three layers:
\begin{itemize}
  \item \textbf{Slides (what the audience sees):} minimal words and clean visuals.
  \item \textbf{Script (what you say):} the short paragraphs under each slide below.
  \item \textbf{Deeper notes (if asked):} extra detail for Q\&A or if you need to explain something differently.
\end{itemize}

\section{Connecting to Class: Physical Electronics Context}
This presentation is a direct application of your \textbf{Physical Electronics} curriculum. Use these connections to ground your talk in what the class has learned.

\subsection*{1. Semiconductor Physics to The MOSFET (Slide 2)}
\begin{itemize}
    \item \textbf{Class Topic:} Energy Bands and Carrier Transport.
    \item \textbf{The Physics:} In a p-type substrate, electrons are minority carriers. When we apply a positive Gate Voltage ($V_G$), the electric field pushes away the holes and pulls energy bands down.
    \item \textbf{Band Bending:} When the bands bend enough, the intrinsic energy level $E_i$ crosses the Fermi level $E_F$. This creates an \textbf{inversion layer} of electrons at the surface. This layer is the conducting "channel."
    \item \textbf{Say this:} ``We studied how doping creates carriers. In a MOSFET, the gate field bends the energy bands to create an inversion layer, turning an insulator into a conductor.''
\end{itemize}

\subsection*{2. Quantitative Analysis to Square Law (Slide 3)}
\begin{itemize}
    \item \textbf{Class Topic:} Drift Current.
    \item \textbf{The Equation:} We know that current density is driven by the electric field:
    \[ J = q \mu n E \]
    Where $q$ is charge, $\mu$ is mobility, $n$ is carrier concentration, and $E$ is the electric field.
    \item \textbf{The Link:} To find the total MOSFET current ($I_D$), we integrate this drift equation along the channel. The result is the famous Square Law for saturation:
    \[ I_D = \frac{1}{2} \mu C_{ox} \frac{W}{L} (V_{GS} - V_{TH})^2 \]
    \item \textbf{Say this:} ``The Square Law isn't just a formula we memorized; it comes directly from integrating the drift current equation ($J = q \mu n E$) across the channel geometry.''
\end{itemize}

\subsection*{3. Device Modeling to Dynamic Power (Slide 7)}
\begin{itemize}
    \item \textbf{Class Topic:} Capacitance and Energy Storage.
    \item \textbf{The Equation:} A capacitor stores energy in its electric field. The energy required to charge a capacitor $C$ to voltage $V$ is:
    \[ E_{stored} = \frac{1}{2} C V^2 \]
    \item \textbf{The Link:} Every digital 1 to 0 transition involves discharging this energy, and every 0 to 1 transition involves charging it again from the power supply.
    \item \textbf{The Physics of $C_{load}$:} The load capacitance comes from the physical gate structure ($C = \epsilon A / d$) and the wires connecting them.
    \item \textbf{Say this:} ``Every logic gate input is physically a capacitor ($C = \epsilon A / d$). Charging it takes energy equal to $\frac{1}{2}CV^2$. That is why voltage is squared in the power equation.''
\end{itemize}

\newpage

\section{Presentation setup (HTML deck)}
\textbf{Your slide file:} \texttt{arm\_presentation.html}

\subsection{How to present it}
\begin{enumerate}
  \item Open \texttt{arm\_presentation.html} in Chrome or Safari.
  \item Use \textbf{Arrow keys} (or Space) to move forward/back.
  \item Press \textbf{F} for fullscreen.
  \item Press \textbf{S} to open speaker view.
\end{enumerate}

\subsection{Timing plan (typical 8--10 minutes)}
\begin{center}
\begin{tabular}{@{}ll@{}}
\toprule
Slide & Suggested time \\
\midrule
1 (Title) & 0:15--0:25 \\
2 (MOSFET) & 0:40--0:60 \\
3 (Current sets speed) & 0:45--1:05 \\
4 (CMOS logic) & 0:35--0:55 \\
5 (CISC versus RISC) & 0:45--1:05 \\
6 (RISC reduces overhead) & 0:45--1:05 \\
7 (Dynamic power) & 0:45--1:10 \\
8 (Pipelining) & 0:45--1:10 \\
9 (Delay versus voltage) & 0:30--0:50 (optional) \\
10 (Area and cost) & 0:30--0:50 \\
11 (SoC) & 0:35--0:55 \\
12 (Conclusion) & 0:15--0:25 \\
\bottomrule
\end{tabular}
\end{center}

\section{Deep Dive: How It All Connects (The Detailed Logic)}
\label{sec:deepdive}
This section connects every concept in a single logical chain.

\subsection*{Step 1: The Physics Constraint}
\textbf{The Reality:} We build computers out of transistors (MOSFETs). A MOSFET is a switch that charges a capacitor (the gate of the next transistor).
\begin{itemize}
    \item \textbf{Capacitance ($C$):} The gate is a parallel plate capacitor. Its capacitance is determined by the oxide thickness ($d$) and the area ($A = W \times L$):
    \[ C_{ox} = \frac{\epsilon_{ox} A}{d} \]
    \item \textbf{Speed depends on Current ($I$):} To switch a logic state (0 to 1), we must charge this capacitor. The time it takes ($\Delta t$) depends on how much current we can push:
    \[ I = C \frac{dV}{dt} \implies \Delta t \approx \frac{C \Delta V}{I} \]
    To switch faster (small $\Delta t$), we need high Current ($I$).
\end{itemize}

\subsection*{Step 2: The Efficiency Conflict}
To get that high current ($I$), the Square Law tells us we need:
\begin{enumerate}
    \item \textbf{Wider Transistors ($W$):} This increases $I$, but it also increases Area ($A$) and therefore Capacitance ($C$). It is a diminishing return.
    \item \textbf{Higher Voltage ($V_{GS}$):} This increases $I$ quadratically, which is great for speed. \textbf{But...}
\end{enumerate}
Power depends on voltage squared ($V^2$). So pushing voltage for speed kills your efficiency.

\subsection*{Step 3: The Architectural Solution (RISC)}
\textbf{The Problem:} x86 (CISC) instructions are complex. Interpreting them requires a large, complex decoder circuit.
\begin{itemize}
    \item \textbf{More Transistors:} A complex decoder needs thousands more logic gates.
    \item \textbf{Higher $C_{load}$:} Every extra gate adds capacitance that must be charged and discharged on every clock cycle.
\end{itemize}
\textbf{The ARM Solution:} Use simple, fixed-length instructions (RISC). The decoder is tiny.
\begin{itemize}
    \item \textbf{Lower $C_{load}$:} Fewer transistors switching means less total capacitance to charge.
\end{itemize}

\subsection*{Step 4: The Efficiency Multiplier}
Because ARM chips have lower $C_{load}$ (less overhead), they waste less power. This allows designers to lower the Voltage ($V_{DD}$) significantly.
\begin{itemize}
    \item The Dynamic Power Equation:
    \[ P_{dynamic} = \alpha \cdot C_{load} \cdot V_{DD}^2 \cdot f \]
\end{itemize}
This is where the magic happens. By reducing $C_{load}$ (Architecture), we enable lower $V_{DD}$ (Physics). Since $V$ is squared, a small drop in voltage creates a massive drop in power consumption.

\newpage

\section{Slide-by-slide Script}

\Slide{1 --- Title: The ARM Advantage}
\Say\\
Today I\textquotesingle m explaining ARM from the bottom up: how the physics of transistors sets constraints, and how ARM\textquotesingle s RISC design is one way to organize silicon to win on performance per watt.

\Slide{2 --- The MOSFET}
\Say\\
Every CPU starts with the MOSFET. It is a voltage-controlled switch. Physically, when we apply a gate voltage, we bend the energy bands of the semiconductor to create an inversion layer---a conductive channel between source and drain.

\Slide{3 --- Current sets speed}
\Say\\
Speed comes from current. We know from class that drift current is $J = q \mu n E$. When we integrate that across the device, we get the Square Law. This tells us that to go faster, we either need bigger transistors (geometry) or more voltage (overdrive).

\Slide{4 --- CMOS logic}
\Say\\
We build logic using CMOS: PMOS pulls up, NMOS pulls down. A single NAND gate takes 4 transistors. A CPU is just billions of these gates arranged to do math.

\Slide{5 --- CISC versus RISC}
\Say\\
CISC (x86) uses complex, variable-length instructions. RISC (ARM) uses simple, uniform instructions. Physically, complex instructions require more hardware to decode, which means more transistors switching.

\Slide{6 --- RISC reduces overhead}
\Say\\
This brings us to $C_{load}$. $C_{load}$ is the total capacitance we have to charge. By simplifying the decoder, ARM reduces the number of transistors that switch, which lowers $C_{load}$.

\Slide{7 --- Dynamic Power}
\Say\\
This is the most important equation: $P = C V^2 f$. This comes from the energy of a capacitor: $E = \frac{1}{2}CV^2$. Because ARM lowers $C$, it can often run at lower $V$. Since $V$ is squared, this results in massive power savings.

\Slide{8 --- Pipelining}
\Say\\
To keep speed high without burning power, ARM uses pipelining. Simple instructions allow for a smooth assembly line, keeping the Cycles Per Instruction (CPI) close to 1.

\Slide{9 --- Delay versus Voltage}
\Say\\
There is a trade-off. Lower voltage reduces the drive current, which slows down the charging of the gate capacitance. ARM finds the "sweet spot" where efficiency is maximized even if peak frequency is slightly lower.

\Slide{10 --- Area and Cost}
\Say\\
Simpler cores use less silicon area. Since cost scales with area squared, simpler cores are much cheaper.

\Slide{11 --- System-on-Chip}
\Say\\
Because the cores are small, we can fit the CPU, GPU, and NPU on one chip. This integration shortens the wires between components. Shorter wires mean less capacitance ($C = \epsilon A / d$), saving even more power.

\Slide{12 --- Conclusion}
\Say\\
Physics dictates that Power scales with Voltage squared. Architecture allows ARM to minimize capacitance ($C_{load}$). The result is a highly efficient chip that powers the mobile world.

\end{document}
