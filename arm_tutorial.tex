\documentclass[12pt]{article}

\usepackage[margin=1in]{geometry}
\usepackage[T1]{fontenc}
\usepackage{lmodern}
\usepackage{microtype}
\usepackage{hyperref}
\usepackage{amsmath,amssymb}
\usepackage{xcolor}
\usepackage{booktabs}

\setlength{\parskip}{8pt}
\setlength{\parindent}{0pt}

\hypersetup{
  colorlinks=true,
  linkcolor=blue,
  urlcolor=blue,
  citecolor=blue
}

% --- Small helpers for readability ---
\newcommand{\Slide}[1]{\subsection*{Slide #1}}
\newcommand{\Goal}[1]{\textbf{Goal:} #1\\}
\newcommand{\PointAt}[1]{\textbf{Point at:} #1\\}
\newcommand{\Say}{\textbf{Say (script):}}
\newcommand{\IfAsked}{\textbf{If asked (deeper):}}
\newcommand{\Transition}[1]{\textbf{Transition line:} #1}
\newcommand{\KeyIdea}[1]{\textbf{Key idea:} #1}

\title{The ARM Advantage\\\large Printable Presenter Tutorial (Detailed \& Expanded)}
\author{Engineering 73}
\date{\today}

\begin{document}
\maketitle
\tableofcontents
\newpage

\section{How to use this tutorial}
This document is designed so you can (1) truly understand the material, (2) rehearse efficiently, and (3) handle Q\&A confidently.

Use three layers:
\begin{itemize}
  \item \textbf{Slides (what the audience sees):} minimal words and clean visuals.
  \item \textbf{Script (what you say):} the short paragraphs under each slide below.
  \item \textbf{Deeper notes (if asked):} extra detail for Q\&A or if you need to explain something differently.
\end{itemize}

\section{Connecting to Class: Physical Electronics Context}
This presentation isn't just about computer chips; it is a direct application of the \textbf{Physical Electronics} curriculum. Use these connections to ground your talk in what the class has already learned.

\subsection*{1. Semiconductor Physics to The MOSFET (Slide 2)}
\begin{itemize}
    \item \textbf{Class Topic:} Energy Bands and Carrier Transport.
    \item \textbf{The Physics Detail:} When we apply a positive Gate Voltage ($V_G$) to an NMOS, we push away the majority holes and attract minority electrons to the surface. This bends the energy bands (conduction band $E_C$ moves closer to the Fermi level $E_F$). Once the bands bend enough (at $V_{TH}$), an \textbf{inversion layer} forms. This layer is the ``channel'' that allows current to flow.
    \item \textbf{Say this:} ``We studied how doping creates carriers. In a MOSFET, the gate field explicitly manipulates those bands to create a conductive channel out of P-type silicon.''
\end{itemize}

\subsection*{2. Quantitative Analysis to Square Law (Slide 3)}
\begin{itemize}
    \item \textbf{Class Topic:} Drift Current ($J = q \mu n E$).
    \item \textbf{The Physics Detail:} Current is Charge $\times$ Velocity. 
    \begin{itemize}
        \item The \textbf{Charge density} in the channel depends on the gate voltage: $Q = C_{ox}(V_{GS} - V_{TH})$.
        \item The \textbf{Velocity} of carriers depends on the electric field (which comes from the drain voltage).
        \item When you multiply Charge ($V$) $\times$ Velocity ($V$), you get Current $\propto V^2$. This is the origin of the square law.
    \end{itemize}
    \item \textbf{Say this:} ``We learned to calculate drift current. The MOSFET square law is just that drift equation integrated across the channel length.''
\end{itemize}

\subsection*{3. Device Modeling to Dynamic Power (Slide 7)}
\begin{itemize}
    \item \textbf{Class Topic:} Capacitance ($C = \epsilon A / d$) and Energy.
    \item \textbf{The Physics Detail:} A digital 1 is just a capacitor charged to $V_{DD}$. A digital 0 is that capacitor discharged to Ground. The energy stored in a capacitor is $E = \frac{1}{2}CV^2$. Every clock cycle, we charge and discharge millions of these capacitors (gates and wires). Therefore, Power = Energy $\times$ Frequency $\approx CV^2f$.
    \item \textbf{Say this:} ``Every logic gate input looks like a capacitor to the previous stage. Charging that capacitor takes energy. That is exactly why $V^2$ dominates the power equation.''
\end{itemize}

\newpage

\section{Presentation setup (HTML deck)}
\textbf{Your slide file:} \texttt{arm\_presentation.html}

\subsection{How to present it}
\begin{enumerate}
  \item Open \texttt{arm\_presentation.html} in Chrome or Safari.
  \item Use \textbf{Arrow keys} (or Space) to move forward/back.
  \item Press \textbf{F} for fullscreen. Press \textbf{Esc} to exit.
  \item Press \textbf{S} to open speaker view.
  \item Have a stable internet connection (MathJax loads from the web).
\end{enumerate}

\section{Slide-by-slide: Detailed Breakdown}

\Slide{1 --- Title: The ARM Advantage}
\Goal{Set the thesis: Physics $\to$ Architecture $\to$ Efficiency.}
\Say\\
Today I\textquotesingle m explaining ARM from the bottom up. We often treat chips as black boxes, but their efficiency comes from specific physical choices. My thesis is that ARM's simpler architecture reduces the number of physical switches needed for "overhead" tasks, which lowers capacitance and enables lower voltage. Since power scales with voltage squared, this creates a massive efficiency advantage.

\Slide{2 --- The MOSFET}
\Goal{Explain the switch.}
\Say\\
Every CPU starts with the MOSFET. It is a voltage-controlled switch. Physically, we have a Gate (metal), an Oxide (insulator), and a Semiconductor (silicon). When we apply voltage to the Gate, it creates an electric field that pulls electrons to the surface, creating a conductive "channel" between Source and Drain.
\IfAsked\\
\textbf{Detail:} The "Field-Effect" in MOSFET refers to this electric field controlling the conductivity.

\Slide{3 --- Current sets speed}
\Goal{Explain $I_D \propto (V_{GS}-V_{TH})^2$.}
\Say\\
Speed comes from current. Why? Because the next transistor is a capacitor. To turn it on, we must fill it with charge. High current fills it faster.
The equation shows we have two knobs to get more current:
1. \textbf{Geometry ($W/L$)}: Make the transistor wider (like a wider highway).
2. \textbf{Overdrive ($V_{GS}-V_{TH}$)}: Push the gas pedal harder (higher voltage).
Both have a cost: Width takes space, Voltage takes power.

\Slide{4 --- CMOS logic}
\Goal{Explain Logic Gates.}
\Say\\
We don't compute with single transistors; we use pairs. CMOS stands for Complementary MOS. We use NMOS to pull signals down to 0, and PMOS to pull them up to 1. A single NAND gate (which can build any logic) takes 4 transistors. A CPU is just billions of these arranged specifically.

\Slide{5 --- CISC versus RISC}
\Goal{The Architectural Divergence.}
\Say\\
Here is where Physics meets Architecture.
\textbf{CISC (x86)} uses complex, variable-length instructions. A single instruction might look like "Load data, add 5, store it back". The hardware to decode this is massive.
\textbf{RISC (ARM)} uses simple, fixed instructions. "Load". "Add". "Store". The hardware is simple.
Physically, "Complexity" means more transistors, more wires, and more switching.

\Slide{6 --- RISC reduces overhead}
\Goal{Explain $C_{load}$.}
\Say\\
Why does complexity matter? Because every extra transistor adds \textbf{Capacitance ($C_{load}$)}.
In x86, the "Decoder" (the part that figures out what to do) is huge because it has to figure out where one instruction ends and the next begins.
In ARM, instructions are fixed size (e.g., always 32 bits). The decoder is tiny.
Less hardware = Less Capacitance = Less Energy per cycle.

\Slide{7 --- Dynamic Power}
\Goal{The Master Equation: $P \approx CV^2f$.}
\Say\\
This is the most important slide. Dynamic Power is:
1. \textbf{Capacitance ($C$)}: Reduced by ARM's simpler logic.
2. \textbf{Frequency ($f$)}: How fast we clock it.
3. \textbf{Voltage ($V^2$)}: This is the killer term. Because it is squared, a small drop in voltage gives a huge drop in power.
Because ARM chips have lower $C$, they run cooler, which allows engineers to lower $V$ even further without crashing the chip.

\Slide{8 --- Pipelining}
\Goal{Speed via Structure.}
\Say\\
If we lower power, how do we stay fast? We use a Pipeline—an assembly line for instructions (Fetch, Decode, Execute).
ARM's simple instructions make this easy. Every stage takes roughly the same amount of time.
x86's complex instructions make this hard (some take 1 cycle, some take 100).
Smooth flow allows ARM to get high performance (Throughput) without brute-forcing the frequency.

\Slide{9 --- Delay versus Voltage}
\Goal{The Trade-off.}
\Say\\
There is no free lunch. If we lower Voltage ($V_{DD}$) to save power, the physics of the MOSFET says the current ($I_D$) drops. Less current means slower switching (Delay increases).
ARM accepts this trade-off. They might run at 3 GHz instead of 5 GHz, but they use 1/3rd the power. For a phone, that is the winning strategy.

\Slide{10 --- Area and Cost}
\Goal{Economic Reality.}
\Say\\
Simpler cores are physically smaller. Silicon real estate is expensive.
Because ARM cores are tiny, we can fit more of them (8 cores, 16 cores) on a chip, OR we can fit other things...

\Slide{11 --- System-on-Chip (SoC)}
\Goal{Integration.}
\Say\\
This leads to the System-on-Chip. Because the CPU is small and efficient, we can paste the GPU, the NPU (AI), and the Memory Controller right next to it on the same piece of silicon.
\textbf{Physics check:} Short wires have less capacitance. Moving data inside an SoC takes way less energy than moving it across a motherboard.

\Slide{12 --- Conclusion}
\Say\\
To summarize:
1. \textbf{Physics:} Voltage squared dominates power.
2. \textbf{Architecture:} RISC reduces the physical transistor count ($C_{load}$) for overhead tasks.
3. \textbf{Result:} This allows lower voltage operation, leading to massive efficiency gains and tighter integration.
ARM isn't just code; it's a strategy for organizing silicon atoms.

\newpage

\section{Deep Dive: The Logical Chain (Detailed)}
Use this section if you get stuck or need to answer a hard question.

\subsection*{The "Why" Chain}
\begin{enumerate}
    \item \textbf{Why does x86 have high capacitance?}
    \begin{itemize}
        \item x86 instructions are variable length (1 byte to 15 bytes).
        \item The CPU doesn't know where the next instruction starts until it decodes the current one.
        \item This requires a "Brute Force" decoder that looks at many bytes at once to find boundaries.
        \item That huge block of logic has thousands of transistors switching every cycle. Switching = Charging Capacitors = Energy.
    \end{itemize}

    \item \textbf{Why does ARM allow lower Voltage?}
    \begin{itemize}
        \item Power generates heat. If a chip gets too hot, silicon degrades or errors occur.
        \item Since ARM has lower Capacitance ($C$), it generates less heat at a given Voltage.
        \item This gives engineers "thermal headroom" to tune the voltage. They can choose to lower $V$ to save massive battery life (since $P \propto V^2$).
    \end{itemize}

    \item \textbf{Why is $V^2$ so powerful?}
    \begin{itemize}
        \item If you drop voltage by 20\% (e.g., 1.0V to 0.8V)...
        \item Power becomes $0.8 \times 0.8 = 0.64$.
        \item You save 36\% of the power for just a 20\% drop in voltage.
    \end{itemize}
\end{enumerate}

\end{document}
